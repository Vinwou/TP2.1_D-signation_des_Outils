%%%%%%%%%%%%%%%%%%%%%%%%%%%%%%%%%%%%%%%%%%%%%%%%%%%%%%%%%%%%%%%%%%%%%
% LaTeX Template: Project Titlepage Modified (v 0.1) by rcx
%
% Original Source: http://www.howtotex.com
% Date: February 2014
% 
% This is a title page template which be used for articles & reports.
% 
% This is the modified version of the original Latex template from
% aforementioned website.
% 
%%%%%%%%%%%%%%%%%%%%%%%%%%%%%%%%%%%%%%%%%%%%%%%%%%%%%%%%%%%%%%%%%%%%%%

\documentclass[12pt]{article}
\usepackage[a4paper]{geometry}
\usepackage[myheadings]{fullpage}
\usepackage{fancyhdr}
\usepackage{lastpage}
\usepackage{graphicx, wrapfig, subcaption, setspace, booktabs}
\usepackage[utf8]{inputenc}
\usepackage[T1]{fontenc}
\usepackage[font=small, labelfont=bf]{caption}
\usepackage{fourier}
\usepackage[protrusion=true, expansion=true]{microtype}
\usepackage[french]{babel}
\usepackage{caption}
\usepackage{sectsty}
\usepackage{url, lipsum}
\usepackage{amsmath}
\usepackage{amssymb}
\usepackage{enumerate}



\usepackage{tikz}
\usepackage[most]{tcolorbox}
\usepackage[pstricks]{bclogo}
\usepackage{pst-blur}




\newcommand{\HRule}[1]{\rule{\linewidth}{#1}}
\onehalfspacing
\setcounter{tocdepth}{5}
\setcounter{secnumdepth}{5}

%-------------------------------------------------------------------------------
% HEADER & FOOTER
%-------------------------------------------------------------------------------
\pagestyle{fancy}
\fancyhf{}
\setlength\headheight{15pt}
\fancyhead[L]{BTS CPRP 1}
\fancyhead[R]{Lycée Le Corbusier}
\fancyfoot[R]{Page \thepage\ sur \pageref{LastPage}}
\fancyfoot[L]{TP - Industrialisation}
%-------------------------------------------------------------------------------
% TITLE PAGE
%-------------------------------------------------------------------------------
 
 

%%%%POUR FAIRE DES EXERCICES INDÉPENDAMMENT DES SECTIONS%%%%
%%%%%%%%%%%%%%%%%%%%%%%%%%%%%%%%%%%%%%%%%%%%%%%%%%%%%%%%%%%%%%%%%%%
\newcounter{exo}
\newenvironment{exo}{\stepcounter{exo}\vspace{0.5cm}{\bfseries Question \theexo\ :}}{\par\vspace{0.5cm}}
%%%%%%%%%%%%%%%%%%%%%%%%%%%%%%%%%%%%%%%%%%%%%%%%%%%%%%%%%%%%%%%%%%%%




\begin{document}
 
\title{ \includegraphics[width=0.18\linewidth]{Images/corbu.jpg} \hspace{2cm} \normalsize \textsc{TP Industrialisation noté \hspace{2cm} \includegraphics[width=0.2\linewidth]{Images/logo.png}}
		\\ [2.0cm]
		\HRule{0.5pt} DOSSIER SUJET \\
		\LARGE \textbf{\uppercase{TP2.1 Désignation des outils de production}}
		\HRule{2pt} \\ [0.5cm]}
\maketitle

\textbf{Deux personnes par groupe}\\
\begin{center}
Toutes les questions seront traitées dans le documents réponse fourni
\end{center}










%-------------------------------------------------------------------------------
% Section title formatting
\sectionfont{\scshape}




\newpage



%%%%%%%%%%%%%%%%%%%%%%%%%%%%%%%%%%%%%%%%%%%%%%%%%%%%%%%%%%%%%%%%
%%%%%%%%%%%%%%%% MACHINE 1 %%%%%%%%%%%%%%%%%%%%%%%%%%%%%%%%%%%%%
%%%%%%%%%%%%%%%%%%%%%%%%%%%%%%%%%%%%%%%%%%%%%%%%%%%%%%%%%%%%%%%%

\tableofcontents
\newpage



\section{Conseils pour le TP}
  \bcinfo Comme pour tous les TPs, il est conseillé de lire toutes les pages une première fois, comme pour un sujet de BTS. Vous noterez que beaucoup d'informations sont dans les annexes (comme au BTS). Certaines définitions et figures sont "en plus". Avant de poser une question, lisez bien tout le dossier. Ne perdez pas de temps, car les TPs sont conçus pour la durée entière : 3h. En plus de comprendre et apprendre, vous devrez écrire vos réponses au propre. Il vous est fortement conseillé de vous partager le travail entre vous deux. Vous ne devez cependant pas communiquer avec d'autres groupes.

\subsection{Matériel}

Crayon, crayons de couleur, stylos, règle et vos cours sont autorisés.

\subsection{Notation}
\noindent
Propreté et clarté dans les réponses : $\pm$3 points\\
Ne répondre que sur le document réponse.\\


\section{Objectifs :}
\begin{center}
\textbf{Découvrir les différents outils nécessaires à la production.}\\
\end{center}

\begin{minipage}[t]{.55\linewidth}
\textit{Compétences transversales travaillées :}
\begin{itemize}
    \item C11 : Définir et mettre en œuvre des essais réels et simulés
\end{itemize}

\end{minipage}
\begin{minipage}[t]{.44\linewidth}
\textit{Savoirs du programme travaillées :}
\begin{itemize}
    \item S5.3 – Conception des outils et porte-outils;
    \item S5.3.3 – Outils;
    \item S7.2.2 – Enlèvement de matière par cisaillement;
    \item S8.1.1 – Élaboration d’avant-projets (outillages retenus);
    \item S8.2 – Paramètres de génération des entités.
\end{itemize}
\end{minipage}

\section{Introduction générale}
\subsection{Process de planification de la production}
%%%%%%%%%%%%%%%%%%%%%%%%%%%%%%%%%%%%%%%%%%%%%%%%%%%%%%%%%%%%%%%%%%
%%%%%%%%%%%%%%%%%%%%%%%%%%%%%%%%%%%%%%%%%%%%%%%%%%%%%%%%%%%%%%%%
\begin{figure}[h]
\centering
\includegraphics[width=0.75\linewidth]{Images/P1.png}
\caption{Au cours du BTS CPRP option sérielle, nous découvrirons tous les étapes des processus pour réaliser une pièce voulue. Ce TP s'insère dans la $3^{ème}$ étape de la figure ci-dessus.}
\label{P1}
\end{figure}
%%%%%%%%%%%%%%%%%%%%%%%%%%%%%%%%%%%%%%%%%%%%%%%%%%%%%%%%%%%%%%%%%%
%%%%%%%%%%%%%%%%%%%%%%%%%%%%%%%%%%%%%%%%%%%%%%%%%%%%%%%%%%%%%%%%%%%


\newpage

\begin{center}



\tikzset{every picture/.style={line width=0.75pt}} %set default line width to 0.75pt        

\begin{tikzpicture}[x=0.75pt,y=0.75pt,yscale=-1,xscale=1]
%uncomment if require: \path (0,300); %set diagram left start at 0, and has height of 300

%Shape: Cube [id:dp3106942406783879] 
\draw   (163,100) -- (183,80) -- (244.55,80) -- (244.55,138) -- (224.55,158) -- (163,158) -- cycle ; \draw   (244.55,80) -- (224.55,100) -- (163,100) ; \draw   (224.55,100) -- (224.55,158) ;
%Shape: Cube [id:dp28620513900019806] 
\draw   (167,190) -- (175,182) -- (236,182) -- (236,240) -- (228,248) -- (167,248) -- cycle ; \draw   (236,182) -- (228,190) -- (167,190) ; \draw   (228,190) -- (228,248) ;
%Shape: Donut [id:dp532385415757469] 
\draw   (186,218) .. controls (186.71,211.92) and (192.22,207) .. (198.29,207) .. controls (204.37,207) and (208.71,211.92) .. (208,218) .. controls (207.29,224.08) and (201.78,229) .. (195.71,229) .. controls (189.63,229) and (185.29,224.08) .. (186,218)(180,218) .. controls (181.1,208.61) and (189.61,201) .. (199,201) .. controls (208.39,201) and (215.1,208.61) .. (214,218) .. controls (212.9,227.39) and (204.39,235) .. (195,235) .. controls (185.61,235) and (178.9,227.39) .. (180,218) ;

%Straight Lines [id:da30920672663549875] 
\draw    (506,50) .. controls (506.5,52.31) and (505.6,53.71) .. (503.29,54.2) .. controls (500.98,54.69) and (500.08,56.09) .. (500.57,58.4) .. controls (501.07,60.71) and (500.17,62.11) .. (497.86,62.6) .. controls (495.55,63.09) and (494.65,64.49) .. (495.15,66.8) .. controls (495.64,69.11) and (494.74,70.51) .. (492.43,71) .. controls (490.12,71.49) and (489.22,72.89) .. (489.72,75.2) .. controls (490.21,77.51) and (489.31,78.91) .. (487,79.4) .. controls (484.69,79.89) and (483.79,81.29) .. (484.29,83.6) .. controls (484.79,85.91) and (483.89,87.31) .. (481.58,87.8) .. controls (479.27,88.29) and (478.37,89.69) .. (478.86,92) .. controls (479.36,94.31) and (478.46,95.71) .. (476.15,96.2) .. controls (473.84,96.69) and (472.94,98.09) .. (473.44,100.4) .. controls (473.93,102.71) and (473.03,104.1) .. (470.72,104.59) -- (469.97,105.76) -- (465.63,112.48) ;
\draw [shift={(464,115)}, rotate = 302.87] [fill={rgb, 255:red, 0; green, 0; blue, 0 }  ][line width=0.08]  [draw opacity=0] (8.93,-4.29) -- (0,0) -- (8.93,4.29) -- cycle    ;
%Straight Lines [id:da36333605403726166] 
\draw    (506,50) .. controls (508.34,49.7) and (509.66,50.72) .. (509.95,53.06) .. controls (510.25,55.4) and (511.57,56.42) .. (513.91,56.12) .. controls (516.25,55.82) and (517.57,56.84) .. (517.86,59.18) .. controls (518.15,61.52) and (519.47,62.54) .. (521.81,62.24) .. controls (524.15,61.94) and (525.47,62.96) .. (525.77,65.3) .. controls (526.06,67.64) and (527.38,68.66) .. (529.72,68.37) .. controls (532.06,68.07) and (533.38,69.09) .. (533.68,71.43) .. controls (533.97,73.77) and (535.29,74.79) .. (537.63,74.49) .. controls (539.97,74.19) and (541.29,75.21) .. (541.58,77.55) .. controls (541.88,79.89) and (543.2,80.91) .. (545.54,80.61) .. controls (547.88,80.31) and (549.2,81.33) .. (549.49,83.67) .. controls (549.78,86.01) and (551.1,87.03) .. (553.44,86.73) .. controls (555.78,86.43) and (557.1,87.45) .. (557.4,89.79) .. controls (557.69,92.13) and (559.01,93.15) .. (561.35,92.85) .. controls (563.69,92.55) and (565.01,93.57) .. (565.3,95.91) .. controls (565.6,98.25) and (566.92,99.27) .. (569.26,98.97) .. controls (571.6,98.67) and (572.92,99.69) .. (573.21,102.03) -- (574.8,103.27) -- (581.13,108.16) ;
\draw [shift={(583.5,110)}, rotate = 217.75] [fill={rgb, 255:red, 0; green, 0; blue, 0 }  ][line width=0.08]  [draw opacity=0] (8.93,-4.29) -- (0,0) -- (8.93,4.29) -- cycle    ;
%Curve Lines [id:da7549614981673118] 
\draw    (282,115) .. controls (284.23,113.65) and (285.92,114.04) .. (287.09,116.19) .. controls (287.72,118.4) and (289.11,119.24) .. (291.26,118.71) .. controls (293.63,118.58) and (294.82,119.72) .. (294.82,122.15) .. controls (294.57,124.52) and (295.57,125.87) .. (297.81,126.2) .. controls (300.1,126.81) and (300.91,128.27) .. (300.24,130.58) .. controls (299.43,132.83) and (300.08,134.33) .. (302.17,135.08) .. controls (304.37,136.31) and (304.91,137.94) .. (303.79,139.97) .. controls (302.59,141.93) and (302.99,143.51) .. (304.98,144.72) .. controls (306.94,146.07) and (307.23,147.73) .. (305.86,149.7) .. controls (304.42,151.6) and (304.61,153.33) .. (306.43,154.89) .. controls (308.18,156.22) and (308.26,157.82) .. (306.66,159.69) .. controls (304.99,161.48) and (304.97,163.11) .. (306.6,164.59) .. controls (308.15,166.55) and (308.03,168.2) .. (306.24,169.55) .. controls (304.37,171.18) and (304.14,172.84) .. (305.57,174.55) .. controls (306.86,176.74) and (306.52,178.4) .. (304.57,179.55) .. controls (302.5,180.96) and (302.05,182.61) .. (303.23,184.52) .. controls (304.34,186.52) and (303.77,188.16) .. (301.53,189.43) -- (301.32,189.97) -- (298.16,196.88) ;
\draw [shift={(297,199)}, rotate = 299.74] [fill={rgb, 255:red, 0; green, 0; blue, 0 }  ][line width=0.08]  [draw opacity=0] (8.93,-4.29) -- (0,0) -- (8.93,4.29) -- cycle    ;
%Curve Lines [id:da4175170921395901] 
\draw    (70,137) .. controls (68.17,135.17) and (68.06,133.38) .. (69.66,131.61) .. controls (71.32,129.96) and (71.34,128.42) .. (69.73,126.99) .. controls (68.2,125.14) and (68.36,123.46) .. (70.22,121.93) .. controls (72.13,120.54) and (72.43,118.85) .. (71.14,116.87) .. controls (69.95,114.75) and (70.36,113.21) .. (72.38,112.24) .. controls (74.51,111.14) and (75.11,109.48) .. (74.18,107.25) .. controls (73.25,105.18) and (73.94,103.68) .. (76.25,102.74) .. controls (78.43,102.2) and (79.26,100.72) .. (78.75,98.31) .. controls (78.22,96.06) and (79.1,94.75) .. (81.39,94.38) .. controls (83.68,94.12) and (84.68,92.84) .. (84.41,90.54) .. controls (84.28,88.17) and (85.41,86.94) .. (87.8,86.83) .. controls (90.17,86.84) and (91.43,85.64) .. (91.58,83.25) .. controls (91.88,80.8) and (93.13,79.77) .. (95.32,80.15) .. controls (97.78,80.4) and (99.14,79.41) .. (99.4,77.19) .. controls (100.11,74.72) and (101.57,73.78) .. (103.8,74.37) .. controls (106,75.02) and (107.4,74.22) .. (108.01,71.99) .. controls (108.74,69.74) and (110.23,68.99) .. (112.5,69.74) .. controls (114.73,70.55) and (116.32,69.85) .. (117.26,67.63) -- (120.39,66.39) -- (127.66,63.86) ;
\draw [shift={(130.5,63)}, rotate = 163.74] [fill={rgb, 255:red, 0; green, 0; blue, 0 }  ][line width=0.08]  [draw opacity=0] (8.93,-4.29) -- (0,0) -- (8.93,4.29) -- cycle    ;
%Shape: Cube [id:dp5589972086691033] 
\draw   (382,118.01) -- (389.97,110.04) -- (414.5,110.04) -- (414.5,133.16) -- (406.53,141.13) -- (382,141.13) -- cycle ; \draw   (414.5,110.04) -- (406.53,118.01) -- (382,118.01) ; \draw   (406.53,118.01) -- (406.53,141.13) ;
%Straight Lines [id:da6766074698451421] 
\draw    (461,141) .. controls (462.78,142.55) and (462.9,144.21) .. (461.35,145.99) .. controls (459.8,147.77) and (459.92,149.43) .. (461.7,150.98) .. controls (463.48,152.52) and (463.6,154.18) .. (462.05,155.96) .. controls (460.5,157.74) and (460.62,159.4) .. (462.4,160.95) .. controls (464.18,162.5) and (464.3,164.16) .. (462.75,165.94) .. controls (461.2,167.72) and (461.32,169.38) .. (463.1,170.93) .. controls (464.88,172.47) and (465,174.13) .. (463.45,175.91) .. controls (461.91,177.69) and (462.03,179.35) .. (463.81,180.9) .. controls (465.59,182.45) and (465.71,184.11) .. (464.16,185.89) .. controls (462.61,187.67) and (462.73,189.33) .. (464.51,190.88) -- (464.73,194.03) -- (465.29,202.01) ;
\draw [shift={(465.5,205)}, rotate = 265.98] [fill={rgb, 255:red, 0; green, 0; blue, 0 }  ][line width=0.08]  [draw opacity=0] (8.93,-4.29) -- (0,0) -- (8.93,4.29) -- cycle    ;
%Shape: Cube [id:dp9274267266045786] 
\draw   (373,207.38) -- (377.64,202.74) -- (413,202.74) -- (413,236.36) -- (408.36,241) -- (373,241) -- cycle ; \draw   (413,202.74) -- (408.36,207.38) -- (373,207.38) ; \draw   (408.36,207.38) -- (408.36,241) ;
%Shape: Donut [id:dp7250755657980528] 
\draw   (384.01,223.61) .. controls (384.43,220.09) and (387.62,217.23) .. (391.14,217.23) .. controls (394.66,217.23) and (397.18,220.09) .. (396.77,223.61) .. controls (396.35,227.13) and (393.16,229.99) .. (389.64,229.99) .. controls (386.12,229.99) and (383.6,227.13) .. (384.01,223.61)(380.54,223.61) .. controls (381.18,218.17) and (386.11,213.75) .. (391.55,213.75) .. controls (396.99,213.75) and (400.89,218.17) .. (400.25,223.61) .. controls (399.61,229.05) and (394.67,233.46) .. (389.23,233.46) .. controls (383.79,233.46) and (379.9,229.05) .. (380.54,223.61) ;

%Curve Lines [id:da07916793730107075] 
\draw    (599,132) .. controls (600.9,133.55) and (601.1,135.27) .. (599.6,137.15) .. controls (598.01,138.82) and (598.08,140.47) .. (599.79,142.1) .. controls (601.46,143.89) and (601.46,145.59) .. (599.77,147.2) .. controls (598.07,148.84) and (598.03,150.55) .. (599.65,152.33) .. controls (601.26,153.92) and (601.21,155.47) .. (599.49,156.98) .. controls (597.76,158.79) and (597.69,160.55) .. (599.3,162.26) .. controls (600.91,163.99) and (600.86,165.68) .. (599.14,167.33) .. controls (597.44,168.73) and (597.41,170.33) .. (599.04,172.12) .. controls (600.69,173.91) and (600.67,175.55) .. (599,177.04) .. controls (597.34,178.87) and (597.36,180.55) .. (599.05,182.06) .. controls (600.76,183.86) and (600.81,185.57) .. (599.2,187.18) .. controls (597.61,188.83) and (597.71,190.56) .. (599.49,192.36) .. controls (601.27,193.81) and (601.4,195.4) .. (599.89,197.13) .. controls (598.4,198.92) and (598.58,200.52) .. (600.43,201.93) -- (600.98,205.79) -- (602.43,213.52) ;
\draw [shift={(603,216)}, rotate = 256.37] [fill={rgb, 255:red, 0; green, 0; blue, 0 }  ][line width=0.08]  [draw opacity=0] (8.93,-4.29) -- (0,0) -- (8.93,4.29) -- cycle    ;
%Curve Lines [id:da3369110243924378] 
\draw  [dash pattern={on 4.5pt off 4.5pt}]  (606,243) .. controls (608.88,259.32) and (604.85,271.95) .. (599.21,285.32) ;
\draw [shift={(598.5,287)}, rotate = 293.2] [color={rgb, 255:red, 0; green, 0; blue, 0 }  ][line width=0.75]    (10.93,-3.29) .. controls (6.95,-1.4) and (3.31,-0.3) .. (0,0) .. controls (3.31,0.3) and (6.95,1.4) .. (10.93,3.29)   ;

% Text Node
\draw  [fill={rgb, 255:red, 155; green, 155; blue, 155 }  ,fill opacity=0.39 ]  (11,145) -- (124,145) -- (124,183) -- (11,183) -- cycle  ;
\draw (14,149) node [anchor=north west][inner sep=0.75pt]  [font=\footnotesize] [align=left] {On a analyser notre\\dessin de définition};
% Text Node
\draw  [fill={rgb, 255:red, 155; green, 155; blue, 155 }  ,fill opacity=0.35 ]  (135,38) -- (270,38) -- (270,76) -- (135,76) -- cycle  ;
\draw (138,42) node [anchor=north west][inner sep=0.75pt]  [font=\footnotesize] [align=left] {\begin{minipage}[lt]{89.36pt}\setlength\topsep{0pt}
On sait quelles sont les surfaces à usiner\end{minipage}};
% Text Node
\draw (252,107) node [anchor=north west][inner sep=0.75pt]  [font=\footnotesize] [align=left] {Brut};
% Text Node
\draw (243,211) node [anchor=north west][inner sep=0.75pt]  [font=\footnotesize] [align=left] {Pièce voulue};
% Text Node
\draw  [fill={rgb, 255:red, 155; green, 155; blue, 155 }  ,fill opacity=0.4 ]  (412.32,10) -- (606.32,10) -- (606.32,48) -- (412.32,48) -- cycle  ;
\draw (509.32,29) node  [font=\footnotesize] [align=left] {\begin{minipage}[lt]{129.29pt}\setlength\topsep{0pt}
En fonction des surfaces à générer sur la pièce voulue\end{minipage}};
% Text Node
\draw    (598.5, 120) circle [x radius= 19.09, y radius= 12.02]   ;
\draw (586,113.5) node [anchor=north west][inner sep=0.75pt]  [font=\footnotesize] [align=left] {Tour};
% Text Node
\draw    (464.5, 127) circle [x radius= 40.31, y radius= 12.02]   ;
\draw (437,120.5) node [anchor=north west][inner sep=0.75pt]  [font=\footnotesize] [align=left] {Fraiseuse};
% Text Node
\draw    (605.5, 229) circle [x radius= 40.31, y radius= 12.02]   ;
\draw (578,222.5) node [anchor=north west][inner sep=0.75pt]  [font=\footnotesize] [align=left] {Fraiseuse};
% Text Node
\draw  [fill={rgb, 255:red, 155; green, 155; blue, 155 }  ,fill opacity=0.4 ]  (624.82,144.5) -- (683.82,144.5) -- (683.82,165.5) -- (624.82,165.5) -- cycle  ;
\draw (654.32,155) node  [font=\footnotesize] [align=left] {Phase 10};
% Text Node
\draw  [fill={rgb, 255:red, 155; green, 155; blue, 155 }  ,fill opacity=0.4 ]  (623.82,107.5) -- (684.82,107.5) -- (684.82,128.5) -- (623.82,128.5) -- cycle  ;
\draw (654.32,118) node  [font=\footnotesize] [align=left] {Exemple :};
% Text Node
\draw  [fill={rgb, 255:red, 155; green, 155; blue, 155 }  ,fill opacity=0.4 ]  (624.82,249.5) -- (683.82,249.5) -- (683.82,270.5) -- (624.82,270.5) -- cycle  ;
\draw (654.32,260) node  [font=\footnotesize] [align=left] {Phase 40};
% Text Node
\draw  [fill={rgb, 255:red, 155; green, 155; blue, 155 }  ,fill opacity=0.4 ]  (624.82,168.5) -- (683.82,168.5) -- (683.82,189.5) -- (624.82,189.5) -- cycle  ;
\draw (654.32,179) node  [font=\footnotesize] [align=left] {Phase 20};
% Text Node
\draw  [fill={rgb, 255:red, 155; green, 155; blue, 155 }  ,fill opacity=0.4 ]  (624.82,194.5) -- (683.82,194.5) -- (683.82,215.5) -- (624.82,215.5) -- cycle  ;
\draw (654.32,205) node  [font=\footnotesize] [align=left] {Phase 30};
% Text Node
\draw  [fill={rgb, 255:red, 155; green, 155; blue, 155 }  ,fill opacity=0.44 ]  (367.82,156.5) -- (426.82,156.5) -- (426.82,177.5) -- (367.82,177.5) -- cycle  ;
\draw (397.32,167) node  [font=\footnotesize] [align=left] {Phase 10};
% Text Node
\draw    (468.5, 220) circle [x radius= 51.62, y radius= 12.02]   ;
\draw (433,213.5) node [anchor=north west][inner sep=0.75pt]  [font=\footnotesize] [align=left] {Pièce voulue};


\end{tikzpicture}
\end{center}


\begin{exo} \textbf{VRAI ou FAUX} (Les réponses s'écrivent toujours sur le document réponse)\end{exo} 
\begin{enumerate}[1)]
    \item En production par enlèvement de matière, le brut à toujours un volume plus faible que la pièce finale.
    \item Si besoin il y a, une pièce peut passer en phase 10 sur un Tour, mais on ne pourra pas l'usiner ensuite sur une fraiseuse.
    \item Si besoin il y a, une pièce peut passer en phase 10 sur un fraiseuse, et on pourra sans problème l'usiner ensuite sur un Tour. 
    \item On ne peut pas réaliser de sphère avec une fraiseuse.
    \item Sur un Tour, l'axe qui sert à commander incrémentalement la rotation est appelé axe $\Vec{C}$.
    \item Lorsqu'on change la position de la pièce lors de son processus de fabrication on change alors de sous-phase.
    \item Le DEC est le vecteur entre l'Origine Porte Pièce et l'Origine Pièce.
    \item Si on réalise un grand lot de pièce, les valeurs du PREF et du DEC ne devraient pas changer entre chaque pièce.
    \item Selon la norme AFNOR NF Z 60-020, sur une MOCN, l'axe $\Vec{X}$ est toujours l'axe parallèle à la broche.
    \item Une mauvaise programmation est dangereuse, en cas de collision à l'intérieur de la machine, le seul moyen de couper le fonctionnement est le bouton d'arrêt d'urgence\footnote{Il est généralement désigné par la dénomination ARU (pour arrêt d'urgence) ou BAU (bouton d'arrêt d'urgence)}.
\end{enumerate}


\begin{exo} Quelle est la première donnée d'entrée dont nous partons toujours pour commencer notre conception de processus ?\end{exo}

\begin{exo} A partir de la pièce de la Figure \ref{bride}, indiquez la matière, la quantité et le type de norme utilisé (Européenne, américaine).\end{exo}





\begin{exo} A partir de la pièce de la Figure \ref{bride}, et d'après vos recherche, quel est le type de traitement utilisé sur la pièce, et quel est son intérêt ? \end{exo}



\section{Quelles sont les différentes opérations d'usinage}
\subsection{Génération de surfaces}
On s'intéresse ici aux surfaces que génèrent les outils sur la pièce. Des outils différents peuvent générer les mêmes surfaces. Les mêmes outils peuvent générer des surfaces différentes en fonction de leur utilisation/programmation sur la machine. La génération des surfaces est exécutée selon une génératrice \footnote{En mathématiques, une génératrice est une figure ou une ligne dont le déplacement engendre une surface. Ces surfaces peuvent être par exemple des surfaces réglées ou de révolution. Plus concrètement, en production la génératrice sera l'arête de l’outil. Voir Figure \ref{directrice} pour plus d'information} et une directrice\footnote{voir figure \ref{directrice} }.

\begin{exo} Quelle est la génératrice qui engendre un \textbf{cylindre} ? Quel est l'\textbf{axe de rotation} pour l'engendrer ? Vous ferez un schéma pour expliquez votre solution.\end{exo}

\begin{exo} Quelle est la génératrice qui engendre un \textbf{cône} ? Quel est l'\textbf{axe de rotation} pour l'engendrer ? Vous ferez un schéma pour expliquez votre solution.\end{exo}


Pour comprendre quelles sont les différentes opérations d'usinage, il nous faut appréhender comment les outils enlèvent de la matière à la pièce usiné.


\begin{minipage}[t]{.55\linewidth}
\includegraphics[width=0.5\linewidth]{Images/C1.JPG}
\end{minipage}
\begin{minipage}[t]{.44\linewidth}
\includegraphics[width=0.7\linewidth]{Images/C2.JPG}
\end{minipage}


\begin{tcolorbox}[colback=blue!5!white,colframe=red!75!black]
  \bcinfo Les livres de production sont à votre disposition dans l'atelier
\end{tcolorbox}


\subsection{Tournage}
\begin{exo} Complétez le tableau d'opération d'usinage en tournage sur le document réponse. Vous devrez renseigner le ou les noms des opérations, la ou les formes/surfaces engendrées et enfin le ou les noms complets des outils.  \end{exo}

\subsection{Fraisage}
\begin{exo} Complétez le tableau d'opération d'usinage en fraisage sur le document réponse. Vous devrez renseigner le ou les noms des opérations, la ou les formes/surfaces engendrées et enfin le ou les noms complet des outils.  \end{exo}






\section{Désignation des outils}
\subsection{Les outils en tournage}
\subsection{Les outils en fraisage}


\begin{exo}\label{exo1} En vous aidant de la Figurindiquez d'où viennent les ordres de déplacement des machines outils.\end{exo}

%%%%%%%%%%%%%%%%%%%%%%%%%%%%%%%%%%%%%%%%%%%%%%%%%%%%%%%%%%%%%%%%%%%%%%%%%%%%%%%%%%%%%%
%%%%%%%%%%%%%%%%%%%%%%%%%%%%%%%%%%%%%%%%%%%%%%%%%%%%%%%%%%%%%%%%%%%%%%%%%%%%%%%%%%%%%%
%%%%%%%%%%%%%%%%%%%%%%%%%%%%%%%%%%%%%%%%%%%%%%%%%%%%%%%%%%%%%%%%%%%%%%%%%%%%%%%%%%%%%%

\begin{tcolorbox}[colback=blue!5!white,colframe=red!75!black]
  \bcinfo szldkjfvhspdjfnmzekjfnmlsqdkjfnmlskdjfnlksmqdjf
\end{tcolorbox}


\section{ANNEXE}
\begin{tcolorbox}[colback=blue!5!white,colframe=orange!75!black]
\begin{center}
    \textbf{Glossaire :}
\end{center}
\begin{itemize}
    \item CNC : Computer Numerical Control (Commande numérique par calculateur);
    \item MOCN : Machine Outil à commande numérique, elle est programmable et équipée d'une "commande numérique par calculateur" (CNC). Elle est en général dédiée à des fabrication variées de pièces différentes, lancées en petits lots répétitifs.
    \item CU : Centre d'usinage : En fait, c'est une MOCN qui est complétée d'autres équipement périphériques qui assurent notamment : \begin{itemize}
        \item le changement automatique d'outils stockés dans des magasins d'outils;
        \item le changement automatique de pièce (palettisation)
        \item éventuellement le convoyage des copeaux (convoyeur)
    \end{itemize}
    Il est dédié à des fabrications variées de pièces différentes.
    \item CAO : Conception Assistée par Ordinateur;
    \item FAO : Fabrication Assistée par Ordinateur.
\end{itemize}
\end{tcolorbox}


\begin{figure}
\centering
\includegraphics[width=0.9\linewidth]{Images/dessin1.JPG}
\caption{Dessin de définition d'une bride de serrage}
\label{bride}
\end{figure}




\begin{figure}
\centering
\includegraphics[width=0.9\linewidth]{Images/gps.JPG}
\caption{Aide mémoire pour spécification géométrique}
\label{gps}
\end{figure}


\begin{figure}
\centering
\includegraphics[width=0.9\linewidth]{Images/directrice.png}
\caption{Directrice : Ligne simple fermée sur laquelle s’appuie une droite mobile, appelée génératrice, en engendrant une surface. La génératrice engendre une surface conique, si cette directrice est une ligne courbe fermée (dessin en bas à gauche), ou bien, engendre une surface pyramidale, si cette directrice est un polygone (dessin en bas à droite).}
\label{directrice}
\end{figure}




\end{document}

